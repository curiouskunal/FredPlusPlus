\documentclass[]{article}

% Imported Packages
%------------------------------------------------------------------------------
\usepackage{amssymb}
\usepackage{amstext}
\usepackage{amsthm}
\usepackage{amsmath}
\usepackage{enumerate}
\usepackage{fancyhdr}
\usepackage[margin=1in]{geometry}
\usepackage{graphicx}
\usepackage{extarrows}
\usepackage{setspace}
\usepackage{gensymb}
%------------------------------------------------------------------------------

% Header and Footer
%------------------------------------------------------------------------------
\pagestyle{plain}  
\renewcommand\headrulewidth{0.4pt}                                      
\renewcommand\footrulewidth{0.4pt}                                    
%------------------------------------------------------------------------------

% Title Details
%------------------------------------------------------------------------------
\title{Fred++\\Deliverable \#1: SRS}
\author{SE 3A04: Software Design II -- Large System Design\\Group 9:\\Kunal 
Shah\\Gabriel Castagner\\Victor Velechovsky\\Josh Mitchell}
\date{}    

\graphicspath{ {images/} }
                           
%------------------------------------------------------------------------------

% Document
%------------------------------------------------------------------------------
\begin{document}

\maketitle	

\section{Introduction}
\label{sec:introduction}
% Begin Section

This section gives a brief introduction of the Software Requirement Specification.

\subsection{Purpose}
\label{sub:purpose}
% Begin SubSection
	The Purpose of this software requirements specification document is to introduce the reader of the proposed idea for the class assignment.
	 This document will also be used to inform the reader of scope of the project, functional and non functional requirements of the software, and any extra dependencies, constraints and references the reader would require. 
	The target audience of this document would be the cliental receiving the final project and any persons whom are designated to implement the design of this product. 
% End SubSection

\subsection{Scope}
\label{sub:scope}
% Begin SubSection
	The project’s name is titled \textit{Fred++}. 
	The project will be app delivered on the Android platform and will be able to let a player interact with a character named Fred.
	Fred will react to the player’s actions based on what is given/ done to him and over time, will alert the player with certain problems that will need to be solved.
	 The project will display its information in a very simplistic, visual fashion, ensuring that the user base will understand everything that is going on and reducing the amount of numbers being displayed on the relatively small screen.\\
	The project will not allow the player to customize Fred due to time constraints of project development. 
	The project will also not display any information in a numerical value as this ensures simplicity and promotes standardization throughout the program development. 
	The project will not require any personal information from the user and thus will not need any sort of security measures to protect its contents.
	Finally the project will not run when the project is not open.\\
	The application of the software is to teach users the affects of health choices made on the bases of consumption and daily routines.
	It will do this by showing how each choice affects each portion of major body systems, see Definitions for these systems.
	The player will be able to inspect these systems further and see how their subsystems interact with each other. 
	The goal of the game is to keep Fred alive for as long as possible, doing this by making decisions that won’t negatively affect him and by effectively treating issues given to the player.
	The information of what each choice does will be gathered from the McMaster’s Anatomy department various documents to accurately gauge how each choice affects the systems in a numerical sense.
% End SubSection

\subsection{Definitions, Acronyms, and Abbreviations}
\label{sub:definitions_acronyms_and_abbreviations}
% Begin SubSection
	Major Body Systems: The specific body functions that this term refers to are the Cardiovascular (Heart), Respiratory (Lungs), 		Gastrointestinal (Digestive), Locomotor (Musculoskeletal) and Nervous (Nerves and Brain) systems.\\
	Project: The project will use project, application, game, and piece of software interchangeably.
	These words all refer to the actual project that is being developed. 

% End SubSection

\subsection{References}
\label{sub:references}
% Begin SubSection
	The project will use many documents from McMaster’s Anatomy department and will be referenced from the website below.\\
	https://fhs.mcmaster.ca/anatomy/

% End SubSection

\subsection{Overview}
\label{sub:overview}
% Begin SubSection
	The rest of the document will contain the overall description of the project and its functional and non functional requirements respectfully. 
	They will discuss the factors that will affect the project, it’s behaviours that it will present and a background for the respective requirements.
	 It is recommended that readers first inspect Section 2, Overall Description before they move onto the final section, Functional and Non-Functional Requirements.
% End SubSection

% End Section

\section{Overall Description}
\label{sec:overall_description}
% Begin Section

This section describes the general factors that affect the application and its 
requirements.

\subsection{Product Perspective}
\label{sub:product_perspective}
% Begin SubSection
	Fred++ is an Android application and as such its functionality relies on 
	the Android operating system as well as the touch screen input and 
	display/audio output of the hardware (phone, tablet, etc) on which it is 
	running.\\
	This application will be responsible for receiving input from the 
	hardware's touch screen, processing that input as it relates to the current 
	state of the system, and outputting any necessary visual and auditory data 
	to the hardware's display and speakers.
% End SubSection

\subsection{Product Functions}
\label{sub:product_functions}
% Begin SubSection
	Fred++ will allow users to interact with a digital character named Fred and 
	influence his health decisions. They will be able to view Fred's various 
	anatomical systems to see how exactly each decision they make is 
	affecting him. In addition, various health metrics will be available to 
	quantitatively measure Fred's well-being. If the user repeatedly chooses 
	poor health choices for Fred, he may suffer adverse effects which grow in 
	severity. These effects may be reversible, and it will be the user's 
	responsibility to make Fred well again.
% End SubSection

\subsection{User Characteristics}
\label{sub:user_characteristics}
% Begin SubSection
	Fred++ is intended to be used by those with a working knowledge of Android 
	device operation, as well as a basic knowledge of healthy vs unhealthy 
	lifestyle decisions (eg. diet \& exercise).\\
	Users need not have a formal education in computer science nor a medical 
	degree to operate this program.
% End SubSection

\subsection{Constraints}
\label{sub:constraints}
% Begin SubSection
	Factors that constrain the development of Fred++ fall into 2 categories: 
	technology and time.\\
	Technological constraints consist of aspects like hardware processing 
	power, maximum screen resolution and battery life. These obstacles must be 
	taken into account, so as to not overextend the practical limitations of 
	the deployment environment.\\
	Time constraints also play a role, as this application must be finished 
	before a pre-determined deadline. As such, some more in-depth features may 
	not be possible to implement in time. 
% End SubSection

\subsection{Assumptions and Dependencies}
\label{sub:assumptions_and_dependencies}
% Begin SubSection
	As Fred++ is an Android application, a reasonable assumption to make is 
	that a sufficiently up-to-date version of the Android operating system will 
	be available on the designated hardware (phone, tablet, etc) on which the 
	application will run.
% End SubSection

\subsection{Apportioning of Requirements}
\label{sub:apportioning_of_requirements}
% Begin SubSection
	Due to aforementioned time constraints, some subtleties within the various 
	ways the user can interact with Fred may need to be delayed until future 
	versions. For example, limiting the number of different foods that are 
	available to feed Fred.
% End SubSection

% End Section

\section{Functional Requirements}

This section outlines the functional requirements of the system, sorted first by viewpoint, and then by business events.
The functions outlined below are assumed to be vital to the proper functionality of the system.

\begin{enumerate}
	\item User Viewpoint
	\begin{enumerate}
	    \item User instantiates application for the first time
	    \begin{enumerate}
	        \item Relevant legal data is displayed, if applicable
	        \item Given the option to start a "new game"
	    \end{enumerate}
	    \item User instantiates application, not for the first time
	    \begin{enumerate}
	        \item User is given the option to either start a new game, or continue an existing one
	    \end{enumerate}
	    \item User starts a new game
	    \begin{enumerate}
	        \item A brand new "subject" is created with randomized "stats"
	        \item The subject is displayed on the screen
	        \item Options for input stimuli are displayed
	    \end{enumerate}
	    \item User stimulates a subsystem
	    \begin{enumerate}
	        \item The subsystem changes its state based on the effects given by the stimulus
	        \item The subsystem notifies all other subsystems of the changes made to its state, and the details of the stimulus that was given
	        \item The other subsystems react by changing their state, based on the effects given by the stimulus, and the updated state of other
	        subsystems
	    \end{enumerate}
		\item User interacts with the GUI
		\begin{enumerate}
			\item User interaction with the input stimuli results in a process initiation that propagates through all subsystems.
			\item User interaction with the input stimuli has a visual indication on the GUI, if applicable, and results in at least one change to the
			state of the subject
			\item User interaction with any objects in the GUI has some visual indication to signify to the user that the interaction was processed
		\end{enumerate}
		\item Time Passes
		\begin{enumerate}
		    \item Changes in state of the subject as a result of time passing produce visual indications given by the GUI, assuming the user is viewing the
		    appropriate subsystem at the appropriate time
		\end{enumerate}
		\item RNG triggers subsystem
		\begin{enumerate}
		    \item If a random number generation produces a visible result, it will be displayed in the GUI, assuming the user is viewing the
		    appropriate subsystem at the appropriate time.
		\end{enumerate}
		\item User selects to view a specific subsystem
		\begin{enumerate}
		    \item When the user selects a different subsystem to be viewed, the GUI is updated to show the details of the state of the given subsystem.
		    \item Detailed information about the state of other subsystems is not displayed and considered irrelevant, with respect to the user.
		\end{enumerate}
	\end{enumerate}
	\item Software Firm Viewpoint
	\begin{enumerate}
	    \item User instantiates application for the first time
	    \begin{enumerate}
	        \item Confirmation of user reading and accepting the legal information is acknowledged and stored
	    \end{enumerate}
	\end{enumerate}
\end{enumerate}

% End Section

\section{Non-Functional Requirements}
\label{sec:non-functional_requirements}
% Begin Section
\subsection{Look and Feel Requirements}
\label{sub:look_and_feel_requirements}
% Begin SubSection

\subsubsection{Appearance Requirements}
\label{ssub:appearance_requirements}
% Begin SubSubSection
\begin{enumerate}[{LF}1. ]
	\item The graphical user interface shall be visually appealing to a minimum of 85\% of users
\end{enumerate}
% End SubSubSection

\subsubsection{Style Requirements}
\label{ssub:style_requirements}
% Begin SubSubSection
\begin{enumerate}[{LF}1. ]
	\item The graphical user interface shall be in color
	\item The graphical user interface shall consist of 2D flat graphics
\end{enumerate}
% End SubSubSection

% End SubSection

\subsection{Usability and Humanity Requirements}
\label{sub:usability_and_humanity_requirements}
% Begin SubSection

\subsubsection{Ease of Use Requirements}
\label{ssub:ease_of_use_requirements}
% Begin SubSubSection
\begin{enumerate}[{UH}1. ]
	\item The product shall be usable by a student in 6th grade or equivalent education
\end{enumerate}
% End SubSubSection

\subsubsection{Personalization and Internationalization Requirements}
\label{ssub:personalization_and_internationalization_requirements}
% Begin SubSubSection
\begin{enumerate}[{UH}1. ]
	\item Application shall not have any region specific language or icons
\end{enumerate}
% End SubSubSection

\subsubsection{Learning Requirements}
\label{ssub:learning_requirements}
% Begin SubSubSection
\begin{enumerate}[{UH}1. ]
	\item The application shall be easy to use by someone who is familiar with operating an android device
	\item A first time user shall not have difficulty understanding what the application does
\end{enumerate}
% End SubSubSection

\subsubsection{Understandability and Politeness Requirements}
\label{ssub:understandability_and_politeness_requirements}
% Begin SubSubSection
\begin{enumerate}[{UH}1. ]
	\item The application should be language ambiguous, limiting written language in favor of graphical icons
\end{enumerate}
% End SubSubSection

\subsubsection{Accessibility Requirements}
\label{ssub:accessibility_requirements}
% Begin SubSubSection
\begin{enumerate}[{UH}1. ]
	\item Application shall be usable by colour blind users.
\end{enumerate}
% End SubSubSection

% End SubSection

\subsection{Performance Requirements}
\label{sub:performance_requirements}
% Begin SubSection

\subsubsection{Speed and Latency Requirements}
\label{ssub:speed_and_latency_requirements}
% Begin SubSubSection
\begin{enumerate}[{PR}1. ]
	\item The application shall load on most "common" 2016 android smart phone devices within 2 seconds of the user clicking the application icon.
	\item The user interface shall not be frozen/hung at any time for more than 3 seconds.
\end{enumerate}
% End SubSubSection

\subsubsection{Safety-Critical Requirements}
\label{ssub:safety_critical_requirements}
% Begin SubSubSection
\begin{enumerate}[{PR}1. ]
	\item The application should not cause the phone overheat to over 100\degree Celsius
	\item The graphics of the application should not cause any physical discomfort in any of its users including seizures. 
\end{enumerate}
% End SubSubSection

\subsubsection{Precision or Accuracy Requirements}
\label{ssub:precision_or_accuracy_requirements}
% Begin SubSubSection
\begin{enumerate}[{PR}1. ]
	\item System shall recognize input with a diameter up to 1 cm. 
\end{enumerate}
% End SubSubSection

\subsubsection{Reliability and Availability Requirements}
\label{ssub:reliability_and_availability_requirements}
% Begin SubSubSection
\begin{enumerate}[{PR}1. ]
	\item After the application is installed on the device, it shall be available for use 24 hours per day, 365 days per year.
\end{enumerate}
% End SubSubSection

\subsubsection{Robustness or Fault-Tolerance Requirements}
\label{ssub:robustness_or_fault_tolerance_requirements}
% Begin SubSubSection
\begin{enumerate}[{PR}1. ]
	\item N/A
\end{enumerate}
% End SubSubSection

\subsubsection{Capacity Requirements}
\label{ssub:capacity_requirements}
% Begin SubSubSection
\begin{enumerate}[{PR}1. ]
	\item N/A
\end{enumerate}
% End SubSubSection

\subsubsection{Scalability or Extensibility Requirements}
\label{ssub:scalability_or_extensibility_requirements}
% Begin SubSubSection
\begin{enumerate}[{PR}1. ]
	\item N/A
\end{enumerate}
% End SubSubSection

\subsubsection{Longevity Requirements}
\label{ssub:longevity_requirements}
% Begin SubSubSection
\begin{enumerate}[{PR}1. ]
	\item The application shall be updated and supported to function on the latest version of the Android operating system for a minimum of one year after the Application is launched
\end{enumerate}
% End SubSubSection

% End SubSection

\subsection{Operational and Environmental Requirements}
\label{sub:operational_and_environmental_requirements}
% Begin SubSection

\subsubsection{Expected Physical Environment}
\label{ssub:expected_physical_environment}
% Begin SubSubSection
\begin{enumerate}[{OE}1. ]
	\item The application shall function in all physical environments the devices the application is running on is able to operate.

\end{enumerate}
% End SubSubSection

\subsubsection{Requirements for Interfacing with Adjacent Systems}
\label{ssub:requirements_for_interfacing_with_adjacent_systems}
% Begin SubSubSection
\begin{enumerate}[{OE}1. ]
	\item The system will run on Android devices with a minimum OS version of Android Marshmallow(6.0)
	\item Application's time will only pass or "tick" when application is running on the device. 
	\item The system shall not require the internet to function
\end{enumerate}
% End SubSubSection

\subsubsection{Productization Requirements}
\label{ssub:productization_requirements}
% Begin SubSubSection
\begin{enumerate}[{OE}1. ]
	\item The application shall be distributed for sale through the Google Play store
	\item The application shall be distributed for free using .APK application file
\end{enumerate}
% End SubSubSection

\subsubsection{Release Requirements}
\label{ssub:release_requirements}
% Begin SubSubSection
\begin{enumerate}[{OE}1. ]
	\item Every application "update" release shall not cause previous features to fail
	\item The application should be ready for public-release by June 1, 2017
\end{enumerate}
% End SubSubSection

% End SubSection

\subsection{Maintainability and Support Requirements}
\label{sub:maintainability_and_support_requirements}
% Begin SubSection

\subsubsection{Maintenance Requirements}
\label{ssub:maintenance_requirements}
% Begin SubSubSection
\begin{enumerate}[{MS}1. ]
	\item The application's documentation shall be relevant and up to date, being updated within one week of any major changes made
\end{enumerate}
% End SubSubSection

\subsubsection{Supportability Requirements}
\label{ssub:supportability_requirements}
% Begin SubSubSection
\begin{enumerate}[{MS}1. ]
	\item There shall be adequate supporting documents of the application after release
\end{enumerate}
% End SubSubSection

\subsubsection{Adaptability Requirements}
\label{ssub:adaptability_requirements}
% Begin SubSubSection
\begin{enumerate}[{MS}1. ]
	\item The application is expected to run on Android Nougat (7.0)
\end{enumerate}
% End SubSubSection

% End SubSection

\subsection{Security Requirements}
\label{sub:security_requirements}
% Begin SubSection

\subsubsection{Access Requirements}
\label{ssub:access_requirements}
% Begin SubSubSection
\begin{enumerate}[{SR}1. ]
	\item N/A
\end{enumerate}
% End SubSubSection

\subsubsection{Integrity Requirements}
\label{ssub:integrity_requirements}
% Begin SubSubSection
\begin{enumerate}[{SR}1. ]
	\item N/A
\end{enumerate}
% End SubSubSection

\subsubsection{Privacy Requirements}
\label{ssub:privacy_requirements}
% Begin SubSubSection
\begin{enumerate}[{SR}1. ]
	\item Application shall not store user's personal information or data
	\item Application shall not transmit user's personal data
\end{enumerate}
% End SubSubSection

\subsubsection{Audit Requirements}
\label{ssub:audit_requirements}
% Begin SubSubSection
\begin{enumerate}[{SR}1. ]
	\item N/A
\end{enumerate}
% End SubSubSection

\subsubsection{Immunity Requirements}
\label{ssub:immunity_requirements}
% Begin SubSubSection
\begin{enumerate}[{SR}1. ]
	\item N/A
\end{enumerate}
% End SubSubSection

% End SubSection

\subsection{Cultural and Political Requirements}
\label{sub:cultural_and_political_requirements}
% Begin SubSection

\subsubsection{Cultural Requirements}
\label{ssub:cultural_requirements}
% Begin SubSubSection
\begin{enumerate}[{CP}1. ]
	\item Application shall not be offensive to religious or ethnic groups
	\item Application shall not display offensive imagery to the user, or contain offensive language
\end{enumerate}
% End SubSubSection

\subsubsection{Political Requirements}
\label{ssub:political_requirements}
% Begin SubSubSection
\begin{enumerate}[{CP}1. ]
	\item N/A
\end{enumerate}
% End SubSubSection

% End SubSection

\subsection{Legal Requirements}
\label{sub:legal_requirements}
% Begin SubSection

\subsubsection{Compliance Requirements}
\label{ssub:compliance_requirements}
% Begin SubSubSection
\begin{enumerate}[{LR}1. ]
	\item Application will abide by all Canadian laws
	\item Application will not infringe on any existing intellectual property or copyright 
\end{enumerate}
% End SubSubSection

\subsubsection{Standards Requirements}
\label{ssub:standards_requirements}
% Begin SubSubSection
\begin{enumerate}[{LR}1. ]
	\item Application will abide by the Google Play Developer Policy standards
\end{enumerate}
% End SubSubSection

% End SubSection

% End Section

\appendix

\newpage
\section{Division of Labour}

%\begin{tabular}{ l c r }
%  1 & 2 & 3 \\
%  4 & 5 & 6 \\
%  7 & 8 & 9 \\
%\end{tabular}

\begin{tabular}{l c r}
    \textbf{Name} & \textbf{Work Completed} & \textbf{Signature} \\
    Kunal Shah & Non-Functional Requirements & 
    \includegraphics[scale=0.2]{Kunal-Sig} \\
    Gabriel Castagner & Introduction & 
    \includegraphics[scale=0.2]{Gabe-Sig} \\
    Victor Velechovsky & Functional Requirements & 
    \includegraphics[scale=0.3]{Vic-Sig} \\
    Josh Mitchell & Overall Description & 
    \includegraphics[scale=0.2]{Josh-Sig} \\
\end{tabular}


\end{document}
%------------------------------------------------------------------------------