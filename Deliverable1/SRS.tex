\documentclass[]{article}

% Imported Packages
%------------------------------------------------------------------------------
\usepackage{amssymb}
\usepackage{amstext}
\usepackage{amsthm}
\usepackage{amsmath}
\usepackage{enumerate}
\usepackage{fancyhdr}
\usepackage[margin=1in]{geometry}
\usepackage{graphicx}
\usepackage{extarrows}
\usepackage{setspace}
%------------------------------------------------------------------------------

% Header and Footer
%------------------------------------------------------------------------------
\pagestyle{plain}  
\renewcommand\headrulewidth{0.4pt}                                      
\renewcommand\footrulewidth{0.4pt}                                    
%------------------------------------------------------------------------------

% Title Details
%------------------------------------------------------------------------------
\title{Fred++\\Deliverable \#1: SRS}
\author{SE 3A04: Software Design II -- Large System Design\\Group 9:\\Kunal 
Shah\\Gabriel Castagner\\Victor Velechovsky\\Josh Mitchell}
\date{}                               
%------------------------------------------------------------------------------

% Document
%------------------------------------------------------------------------------
\begin{document}

\maketitle	

\section{Introduction}
\label{sec:introduction}
% Begin Section

This section of the SRS should provide an overview of the entire SRS.

\subsection{Purpose}
\label{sub:purpose}
% Begin SubSection
\begin{enumerate}[a)]
	\item Delineate the purpose of the SRS
	\item Specify the intended audience for the SRS
\end{enumerate}
% End SubSection

\subsection{Scope}
\label{sub:scope}
% Begin SubSection
\begin{enumerate}[a)]
	\item Identify the software product(s) to be produced by name (e.g., Host DBMS, Report Generator, etc.)
	\item Explain what the software product(s) will, and, if necessary, will not do
	\item Describe the application of the software being specified, including relevant benefits, objectives, and goals
	\item Be consistent with similar statements in higher-level specifications (e.g., the system requirements specification), if they exist
\end{enumerate}
% End SubSection

\subsection{Definitions, Acronyms, and Abbreviations}
\label{sub:definitions_acronyms_and_abbreviations}
% Begin SubSection
\begin{enumerate}[a)]
	\item \textbf{Subject} - The simulated "Human Body" that is the central object in the app.
    \item \textbf{GUI} - Graphical User Interface
    \item \textbf{Human Body Model} - An abstract module that models the behaviour of the human body and its response to various stimuli
    \item \textbf{Input Stimuli} - Food, Drink, Exercise, Medicine
\end{enumerate}
% End SubSection

\subsection{References}
\label{sub:references}
% Begin SubSection
\begin{enumerate}[a)]
	\item Provide a complete list of all documents referenced elsewhere in the SRS
	\item Identify each document by title, report number (if applicable), date, and publishing organization
	\item Specify the sources from which the references can be obtained
\end{enumerate}
% End SubSection

\subsection{Overview}
\label{sub:overview}
% Begin SubSection
\begin{enumerate}[a)]
	\item Describe what the rest of the SRS contains
	\item Explain how the SRS is organized
\end{enumerate}
% End SubSection

% End Section

\section{Overall Description}
\label{sec:overall_description}
% Begin Section

This section describes the general factors that affect the application and its 
requirements.

\subsection{Product Perspective}
\label{sub:product_perspective}
% Begin SubSection
	Fred++ is an Android application and as such its functionality relies on 
	the Android operating system as well as the touch screen input and 
	display/audio output of the hardware (phone, tablet, etc) on which it is 
	running.\\
	This application will be responsible for receiving input from the 
	hardware's touch screen, processing that input as it relates to the current 
	state of the system, and outputting any necessary visual and auditory data 
	to the hardware's display and speakers.
% End SubSection

\subsection{Product Functions}
\label{sub:product_functions}
% Begin SubSection
	Fred++ will allow users to interact with a digital character named Fred and 
	influence his health decisions. They will be able to view Fred's various 
	anatomical systems to see how exactly each decision they make is 
	affecting him. In addition, various health metrics will be available to 
	quantitatively measure Fred's well-being. If the user repeatedly chooses 
	poor health choices for Fred, he may suffer adverse effects which grow in 
	severity. These effects may be reversible, and it will be the user's 
	responsibility to make Fred well again.
% End SubSection

\subsection{User Characteristics}
\label{sub:user_characteristics}
% Begin SubSection
	Fred++ is intended to be used by those with a working knowledge of Android 
	device operation, as well as a basic knowledge of healthy vs unhealthy 
	lifestyle decisions (eg. diet \& exercise).\\
	Users need not have a formal education in computer science nor a medical 
	degree to operate this program.
% End SubSection

\subsection{Constraints}
\label{sub:constraints}
% Begin SubSection
	Factors that constrain the development of Fred++ fall into 2 categories: 
	technology and time.\\
	Technological constraints consist of aspects like hardware processing 
	power, maximum screen resolution and battery life. These obstacles must be 
	taken into account, so as to not overextend the practical limitations of 
	the deployment environment.\\
	Time constraints also play a role, as this application must be finished 
	before a pre-determined deadline. As such, some more in-depth features may 
	not be possible to implement in time. 
% End SubSection

\subsection{Assumptions and Dependencies}
\label{sub:assumptions_and_dependencies}
% Begin SubSection
	As Fred++ is an Android application, a reasonable assumption to make is 
	that a sufficiently up-to-date version of the Android operating system will 
	be available on the designated hardware (phone, tablet, etc) on which the 
	application will run.
% End SubSection

\subsection{Apportioning of Requirements}
\label{sub:apportioning_of_requirements}
% Begin SubSection
	Due to aforementioned time constraints, some subtleties within the various 
	ways the user can interact with Fred may need to be delayed until future 
	versions. For example, limiting the number of different foods that are 
	available to feed Fred.
% End SubSection

% End Section

\section{Functional Requirements}

This section outlines the functional requirements of the system, sorted first by viewpoint, and then by business events.
The functions outlined below are assumed to be vital to the proper functionality of the system.

\begin{enumerate}
	\item User Viewpoint
	\begin{enumerate}
	    \item User instantiates application for the first time
	    \begin{enumerate}
	        \item Relevant legal data is displayed, if applicable
	        \item Given the option to start a "new game"
	    \end{enumerate}
	    \item User instantiates application, not for the first time
	    \begin{enumerate}
	        \item User is given the option to either start a new game, or continue an existing one
	    \end{enumerate}
	    \item User starts a new game
	    \begin{enumerate}
	        \item A brand new "subject" is created with randomized "stats"
	        \item The subject is displayed on the screen
	        \item Options for input stimuli are displayed
	    \end{enumerate}
	    \item User stimulates a subsystem
	    \begin{enumerate}
	        \item The subsystem changes its state based on the effects given by the stimulus
	        \item The subsystem notifies all other subsystems of the changes made to its state, and the details of the stimulus that was given
	        \item The other subsystems react by changing their state, based on the effects given by the stimulus, and the updated state of other
	        subsystems
	    \end{enumerate}
		\item User interacts with the GUI
		\begin{enumerate}
			\item User interaction with the input stimuli results in a process initiation that propagates through all subsystems.
			\item User interaction with the input stimuli has a visual indication on the GUI, if applicable, and results in at least one change to the
			state of the subject
			\item User interaction with any objects in the GUI has some visual indication to signify to the user that the interaction was processed
		\end{enumerate}
		\item Time Passes
		\begin{enumerate}
		    \item Changes in state of the subject as a result of time passing produce visual indications given by the GUI, assuming the user is viewing the
		    appropriate subsystem at the appropriate time
		\end{enumerate}
		\item RNG triggers subsystem
		\begin{enumerate}
		    \item If a random number generation produces a visible result, it will be displayed in the GUI, assuming the user is viewing the
		    appropriate subsystem at the appropriate time.
		\end{enumerate}
		\item User selects to view a specific subsystem
		\begin{enumerate}
		    \item When the user selects a different subsystem to be viewed, the GUI is updated to show the details of the state of the given subsystem.
		    \item Detailed information about the state of other subsystems is not displayed and considered irrelevant, with respect to the user.
		\end{enumerate}
	\end{enumerate}
	\item Legal Viewpoint
	\begin{enumerate}
	    \item User initiates the system for the first time
	    \begin{enumerate}
	        \item User is informed of legal information pertaining to the use of the software, if and where applicable
	    \end{enumerate}
	\end{enumerate}
	\item Human Body Model Viewpoint
	\begin{enumerate}
	    \item User stimulates a subsystem
	    	\begin{enumerate}
        \item The list of subsystems corresponding to the human body's systems which react to the stimulant are given as:
        \begin{enumerate}
            \item Cardiovascular/Respiratory (heart/lungs)
            \item Gastrointestinal (digestive)
            \item Locomotor (musculoskeletal)
            \item Nervous (nerves and brain)
        \end{enumerate}
        \item The list of input stimuli are given as:
        \begin{enumerate}
            \item Food
            \item Drink
            \item Exercise
            \item Medicine
        \end{enumerate}
        \end{enumerate}
        \item User selects to view a specific subsystem
        \begin{enumerate}
           \item The list of metrics tracking the overall status of the subject are given as:
            \begin{enumerate}
                \item Hunger
                \item Thrist
                \item Happiness
                \item Weight
                \item Health
	        \end{enumerate} 
        \end{enumerate}

	\end{enumerate}
	\item Software Firm Viewpoint
	\begin{enumerate}
	    \item User instantiates application for the first time
	    \begin{enumerate}
	        \item Confirmation of user reading and accepting the legal information is acknowledged and stored
	    \end{enumerate}
	\end{enumerate}
\end{enumerate}

% End Section

\section{Non-Functional Requirements}
\label{sec:non-functional_requirements}
% Begin Section
\subsection{Look and Feel Requirements}
\label{sub:look_and_feel_requirements}
% Begin SubSection

\subsubsection{Appearance Requirements}
\label{ssub:appearance_requirements}
% Begin SubSubSection
\begin{enumerate}[{LF}1. ]
	\item 
\end{enumerate}
% End SubSubSection

\subsubsection{Style Requirements}
\label{ssub:style_requirements}
% Begin SubSubSection
\begin{enumerate}[{LF}1. ]
	\item 
\end{enumerate}
% End SubSubSection

% End SubSection

\subsection{Usability and Humanity Requirements}
\label{sub:usability_and_humanity_requirements}
% Begin SubSection

\subsubsection{Ease of Use Requirements}
\label{ssub:ease_of_use_requirements}
% Begin SubSubSection
\begin{enumerate}[{UH}1. ]
	\item 
\end{enumerate}
% End SubSubSection

\subsubsection{Personalization and Internationalization Requirements}
\label{ssub:personalization_and_internationalization_requirements}
% Begin SubSubSection
\begin{enumerate}[{UH}1. ]
	\item 
\end{enumerate}
% End SubSubSection

\subsubsection{Learning Requirements}
\label{ssub:learning_requirements}
% Begin SubSubSection
\begin{enumerate}[{UH}1. ]
	\item 
\end{enumerate}
% End SubSubSection

\subsubsection{Understandability and Politeness Requirements}
\label{ssub:understandability_and_politeness_requirements}
% Begin SubSubSection
\begin{enumerate}[{UH}1. ]
	\item 
\end{enumerate}
% End SubSubSection

\subsubsection{Accessibility Requirements}
\label{ssub:accessibility_requirements}
% Begin SubSubSection
\begin{enumerate}[{UH}1. ]
	\item 
\end{enumerate}
% End SubSubSection

% End SubSection

\subsection{Performance Requirements}
\label{sub:performance_requirements}
% Begin SubSection

\subsubsection{Speed and Latency Requirements}
\label{ssub:speed_and_latency_requirements}
% Begin SubSubSection
\begin{enumerate}[{PR}1. ]
	\item 
\end{enumerate}
% End SubSubSection

\subsubsection{Safety-Critical Requirements}
\label{ssub:safety_critical_requirements}
% Begin SubSubSection
\begin{enumerate}[{PR}1. ]
	\item 
\end{enumerate}
% End SubSubSection

\subsubsection{Precision or Accuracy Requirements}
\label{ssub:precision_or_accuracy_requirements}
% Begin SubSubSection
\begin{enumerate}[{PR}1. ]
	\item 
\end{enumerate}
% End SubSubSection

\subsubsection{Reliability and Availability Requirements}
\label{ssub:reliability_and_availability_requirements}
% Begin SubSubSection
\begin{enumerate}[{PR}1. ]
	\item 
\end{enumerate}
% End SubSubSection

\subsubsection{Robustness or Fault-Tolerance Requirements}
\label{ssub:robustness_or_fault_tolerance_requirements}
% Begin SubSubSection
\begin{enumerate}[{PR}1. ]
	\item 
\end{enumerate}
% End SubSubSection

\subsubsection{Capacity Requirements}
\label{ssub:capacity_requirements}
% Begin SubSubSection
\begin{enumerate}[{PR}1. ]
	\item 
\end{enumerate}
% End SubSubSection

\subsubsection{Scalability or Extensibility Requirements}
\label{ssub:scalability_or_extensibility_requirements}
% Begin SubSubSection
\begin{enumerate}[{PR}1. ]
	\item 
\end{enumerate}
% End SubSubSection

\subsubsection{Longevity Requirements}
\label{ssub:longevity_requirements}
% Begin SubSubSection
\begin{enumerate}[{PR}1. ]
	\item 
\end{enumerate}
% End SubSubSection

% End SubSection

\subsection{Operational and Environmental Requirements}
\label{sub:operational_and_environmental_requirements}
% Begin SubSection

\subsubsection{Expected Physical Environment}
\label{ssub:expected_physical_environment}
% Begin SubSubSection
\begin{enumerate}[{OE}1. ]
	\item 
\end{enumerate}
% End SubSubSection

\subsubsection{Requirements for Interfacing with Adjacent Systems}
\label{ssub:requirements_for_interfacing_with_adjacent_systems}
% Begin SubSubSection
\begin{enumerate}[{OE}1. ]
	\item 
\end{enumerate}
% End SubSubSection

\subsubsection{Productization Requirements}
\label{ssub:productization_requirements}
% Begin SubSubSection
\begin{enumerate}[{OE}1. ]
	\item 
\end{enumerate}
% End SubSubSection

\subsubsection{Release Requirements}
\label{ssub:release_requirements}
% Begin SubSubSection
\begin{enumerate}[{OE}1. ]
	\item 
\end{enumerate}
% End SubSubSection

% End SubSection

\subsection{Maintainability and Support Requirements}
\label{sub:maintainability_and_support_requirements}
% Begin SubSection

\subsubsection{Maintenance Requirements}
\label{ssub:maintenance_requirements}
% Begin SubSubSection
\begin{enumerate}[{MS}1. ]
	\item 
\end{enumerate}
% End SubSubSection

\subsubsection{Supportability Requirements}
\label{ssub:supportability_requirements}
% Begin SubSubSection
\begin{enumerate}[{MS}1. ]
	\item 
\end{enumerate}
% End SubSubSection

\subsubsection{Adaptability Requirements}
\label{ssub:adaptability_requirements}
% Begin SubSubSection
\begin{enumerate}[{MS}1. ]
	\item 
\end{enumerate}
% End SubSubSection

% End SubSection

\subsection{Security Requirements}
\label{sub:security_requirements}
% Begin SubSection

\subsubsection{Access Requirements}
\label{ssub:access_requirements}
% Begin SubSubSection
\begin{enumerate}[{SR}1. ]
	\item 
\end{enumerate}
% End SubSubSection

\subsubsection{Integrity Requirements}
\label{ssub:integrity_requirements}
% Begin SubSubSection
\begin{enumerate}[{SR}1. ]
	\item 
\end{enumerate}
% End SubSubSection

\subsubsection{Privacy Requirements}
\label{ssub:privacy_requirements}
% Begin SubSubSection
\begin{enumerate}[{SR}1. ]
	\item 
\end{enumerate}
% End SubSubSection

\subsubsection{Audit Requirements}
\label{ssub:audit_requirements}
% Begin SubSubSection
\begin{enumerate}[{SR}1. ]
	\item 
\end{enumerate}
% End SubSubSection

\subsubsection{Immunity Requirements}
\label{ssub:immunity_requirements}
% Begin SubSubSection
\begin{enumerate}[{SR}1. ]
	\item 
\end{enumerate}
% End SubSubSection

% End SubSection

\subsection{Cultural and Political Requirements}
\label{sub:cultural_and_political_requirements}
% Begin SubSection

\subsubsection{Cultural Requirements}
\label{ssub:cultural_requirements}
% Begin SubSubSection
\begin{enumerate}[{CP}1. ]
	\item 
\end{enumerate}
% End SubSubSection

\subsubsection{Political Requirements}
\label{ssub:political_requirements}
% Begin SubSubSection
\begin{enumerate}[{CP}1. ]
	\item 
\end{enumerate}
% End SubSubSection

% End SubSection

\subsection{Legal Requirements}
\label{sub:legal_requirements}
% Begin SubSection

\subsubsection{Compliance Requirements}
\label{ssub:compliance_requirements}
% Begin SubSubSection
\begin{enumerate}[{LR}1. ]
	\item 
\end{enumerate}
% End SubSubSection

\subsubsection{Standards Requirements}
\label{ssub:standards_requirements}
% Begin SubSubSection
\begin{enumerate}[{LR}1. ]
	\item 
\end{enumerate}
% End SubSubSection

% End SubSection

% End Section

\appendix
\section{Division of Labour}
\label{sec:division_of_labour}
% Begin Section
Include a Division of Labour sheet which indicates the contributions of each team member. This sheet must be signed by all team members.
% End Section

\newpage
\section*{IMPORTANT NOTES}
\begin{itemize}
	\item Be sure to include all sections of the template in your document regardless whether you have something to write for each or not
	\begin{itemize}
		\item If you do not have anything to write in a section, indicate this by the \emph{N/A}, \emph{void}, \emph{none}, etc.
	\end{itemize}
	\item Uniquely number each of your requirements for easy identification and cross-referencing
	\item Highlight terms that are defined in Section~1.3 (\textbf{Definitions, Acronyms, and Abbreviations}) with \textbf{bold}, \emph{italic} or \underline{underline}
	\item For Deliverable 1, please highlight, in some fashion, all (you may have more than one) creative and innovative features. Your creative and innovative features will generally be described in Section~2.2 (\textbf{Product Functions}), but it will depend on the type of creative or innovative features you are including.
\end{itemize}


\end{document}
%------------------------------------------------------------------------------