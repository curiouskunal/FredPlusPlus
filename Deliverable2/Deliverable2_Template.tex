\documentclass[]{article}

% Imported Packages
%------------------------------------------------------------------------------
\usepackage{amssymb}
\usepackage{amstext}
\usepackage{amsthm}
\usepackage{amsmath}
\usepackage{enumerate}
\usepackage{fancyhdr}
\usepackage[margin=1in]{geometry}
\usepackage{graphicx}
\usepackage{extarrows}
\usepackage{setspace}
\usepackage{parskip}
%------------------------------------------------------------------------------

% Header and Footer
%------------------------------------------------------------------------------
\pagestyle{plain}  
\renewcommand\headrulewidth{0.4pt}                                      
\renewcommand\footrulewidth{0.4pt}                                    
%------------------------------------------------------------------------------

% Title Details
%------------------------------------------------------------------------------
\title{Deliverable \#2}
\author{SE 3A04: Software Design II -- Large System Design}
\date{}                               
%------------------------------------------------------------------------------

% Document
%------------------------------------------------------------------------------
\begin{document}

\maketitle	

\section{Introduction}
\label{sec:introduction}
% Begin Section


\subsection{Purpose}
\label{sub:purpose}
% Begin SubSection

This document shall provide a brief overview of the overall Design Architecture
of FredPlusPlus.\\

The target audience of this document would be the clientele receiving the final
project and any persons designated to implement the design of this product. The
other target audience would be the Teaching Assistants and peers that will
inspect and/or mark this document.
% End SubSection

\subsection{System Description}
\label{sub:system_description}
% Begin SubSection
%\begin{enumerate}[a)]

The system that is mentioned in the document is a system that simulates the anatomy
of the human body. The subsystems that the program will simulate will be the major
bodily systems that were referred to in Deliverable 1 (please see the definitions for
more information). The stimuli that act upon the sub systems will directly affect
other subsystems. For example, if the player fed Fred a fatty food, this would
increase his BMI, which would result in a higher blood pressure. Thus, the stimuli was
applied to the digestive system and eventually resulted in a change in the circulatory
system.
	
We have organized the project by grouping each anotomical system into networks that have
their own dependencies and \textbf{outsourced dependencies} from other networks
that may or may not be \textbf{"active"} in the system. Some networks will have stimuli that
the user can use directly, such as feeding. Some changes to the networksmay occur indirectly
from the player's actions, or from stimuli generated by the system. These include,
for example, a random chance of Fred being hit by a car.

%\end{enumerate}
% End SubSection
\subsection{Definitions}
\label{sub:system_description}
% Begin SubSection

\hspace{5mm}\textbf{Body Systems:} The specific anatomical systems that this document refers to are the:
\begin{enumerate}
	\item Cardiovascular (Heart)
	\item Respiratory (Lungs)
	\item Gastrointestinal (Digestive)
	\item Locomotor (Musculoskeletal)
	\item Nervous (Nerves \& Brain)
\end{enumerate} 

\vspace{5mm}

\textbf{Project:} The project will use project, application, game, and piece of software interchangeably.
These words all refer to the actual project that is being developed.

% End SubSection
\subsection{Overview}
\label{sub:overview}
% Begin SubSection
%\begin{enumerate}[a)]
The rest of the document will contain the corresponding Use Case Diagrams,
Analysis Class Diagrams,the System Architecture and the Class Responsibility Cards.
They will visually demonstrate to the user how the classes interact with one another,
how the implemented architecture will represent the overall architecture,
and the contents of the CRC Cards. It is recommended that readers inspect the
document in sequential order to first understand the layout of the design, and then
further understand the purpose of the CRC Cards.

%This includes Use Case Diagram, Analysis Class Diagram, System Architecture and Class Responsibility Cards.

%\end{enumerate}
% End SubSection

% End Section

\section{Use Case Diagram}
\label{sec:use_case_diagram}
% Begin Section

\begin{figure}[h!]
\includegraphics[width=\linewidth]{../Resources/UseCaseDiagram.png}
\caption{UseCaseDiagram}
\end{figure}

\begin{enumerate}[a)]

	\item \textbf{User wants to turn on a system}.
	The nervous system will be
	'connected' to the system. It will begin affect the other subsystems, as 
	well as the overall	state according to its internal logic. Any stimuli 
	that it includes will now be available to the user. The subsystem will now 
	be viewable. Ex: The user wants to turn on the nervous system. 
	
	\item \textbf{User wants to turn off a system}
	, meaning it would be
	'disconnected' from the system. It will cease to affect the state of any
	of the subsystems, or the system as a whole, in any way. The user will no 
	longer be able to select stimuli that are exclusive to this system. The 
	user will no longer	be able to inspect this system.
	Ex: The user requests to turn off the nervous system
	
	\item \textbf{User wants to view a subsystem}. The subsystem will become 
	visible	in the User Interface. Ex. The user selects to the view the 
	nervous system.
	
	\item \textbf{User wants to hide a subsystem}. The subsystem will become 
	hidden from the user interface.	Ex. The user select to hide the nervous 
	system.
	
	\item \textbf{User wants to give stimulus to the system}. Effects from this 
	stimulus become	active and the user examines the changes in state to Fred's 
	metrics. Ex. The user selects a food stimulus and notices a decrease in 
	Fred's hunger metric.
	
\end{enumerate}

% End Section
\newpage
\section{Analysis Class Diagram}
\label{sec:analysis_class_diagram}
% Begin Section

\begin{figure}[h!]
\includegraphics[width=\linewidth]{../Resources/ActivityClassDiagram_V3.png}
\caption{Activity Class Diagram}
\end{figure}
% End Section


\section{Architectural Design}
\label{sec:architectural_design}
% Begin Section
This section should provide an overview of the overall architectural design of
your application. You overall architecture should show the division of the system
into subsystems with high cohesion and low coupling.

\subsection{System Architecture}
\label{sub:system_architecture}
% Begin SubSection

The overall Architecture that we will be using for this project will be the 
Presentation-Abstraction-Control or PAC design. This architecture��s components 
are decomposed in following sub sections: \\

\textbf{Presentation}: This section is responsible for handling all of the programs��
GUI interfaces, images and user interfaces. It is dependent on the abstraction section
for its specific information, it does not communicate backwards towards the abstraction section. \\

\textbf{Abstraction}: This section is responsible for all the input and output
handling through the use of logic, triggers and data keeping. This section gets its��
inputs/ stimuli from the Controller section and then interprets it and when finished
with it, sends the visualization of that data to the Presentation section. The bulk
of the project will be put in this section. \\

\textbf{Control}: This section is responsible for handling all of the user input
for the program and sending that information to the Abstraction section. It should
be able to handle all the button handling, sets, input values without doing any
calculations to them, only interpreting that info and sending it to the Abstraction to be used. \\

Our program will use this architecture as it encourages high cohesion and low coupling
through the use of modularizing the data and keeping one system responsible for logic calculations.
The expected platform that we hope to implement this project on also encourages this as
Android development platforms promote PAC architecture. Finally, the structure and
organization of our project follows intimately follower the PAC architecture.

% End SubSection

\subsection{Subsystems}
\label{sub:subsystems}
% Begin SubSection

The analysis classes can be logically grouped based on the overall architecture
of the system: presentation, abstraction, and control.

\subsubsection{Presentation}

The presentation subsystem is in charge of formatting and displayed the data given
to it by the rest of the system (namely, the control).

\subsubsection{Abstraction}

The abstraction subsystem contains the majority of the business logic of the
internal model of the system. It is responsible for processing input data and
sending the result to the control.

\subsubsection{Control}

The control subsystem is an intermediary between the presentation and the
abstraction. It acts to control the flow of information throughout the system,
and provides an interface for the abstraction to access the presentation.

% End SubSection

% End Section
	
\section{Class Responsibility Collaboration (CRC) Cards}
\label{sec:class_responsibility_collaboration_crc_cards}
% Begin Section
This section contains all of the Class Responsibility Collaboration cards for 
FredPlusPlus.

	\begin{table}[ht]
		\centering
		\begin{tabular}{|p{5cm}|p{5cm}|}
		\hline 
		 \multicolumn{2}{|l|}{\textbf{Class Name: activity\_main}} \\
		\hline
		\textbf{Responsibility:} & \textbf{Collaborators:} \\
		\hline
		Display Fred, Stimuli \& Subsystems to User & MainActivity\\
		\vspace{0.2in} & \\
		\hline
		Send click event data from 'Toggle Subsystem' buttons & Toggle Subsystem Listener\\
		\vspace{0.2in} & \\
		\hline
		Send click event data from 'View Subsystem' buttons & Show Subsystem Listener \\
		\vspace{0.2in} & \\
		\hline
		Send click event data from 'Stimulus' buttons & Stimuli Listener\\
		\vspace{0.2in} & \\
		\hline
		\end{tabular}
	\end{table}
	%Class 3
	\begin{table}[ht]
		\centering
		\begin{tabular}{|p{5cm}|p{5cm}|}
		\hline 
		 \multicolumn{2}{|l|}{\textbf{Class Name: Stimuli Listener}} \\
		\hline
		\textbf{Responsibility:} & \textbf{Collaborators:} \\
		\hline
		Receive click event data & activity\_main \\
		\vspace{0.2in} & \\
		\hline
		Notify the net result calculator that a stimulus has been clicked & Fred Attribute Net Result Calc \\
		\vspace{0.2in} & \\
		\hline
		\end{tabular}
	\end{table}
	%Class 4
	\begin{table}[ht]
		\centering
		\begin{tabular}{|p{5cm}|p{5cm}|}
			\hline 
			\multicolumn{2}{|l|}{\textbf{Class Name: Toggle Subsystem 
			Listener}} \\
			\hline
			\textbf{Responsibility:} & \textbf{Collaborators:} \\
			\hline
			Listen for Toggle Subsystem taps from the GUI & activity\_main\\
			\vspace{0.2in} & \\
			\hline
			Inform inner system when a Subsystem has been toggled & Fred 
			Attribute Date Store\\ 
			\vspace{0.2in} & \\
			\hline
		\end{tabular}
	\end{table}\
	%Class 5
	\begin{table}[ht]
		\centering
		\begin{tabular}{|p{5cm}|p{5cm}|}
			\hline 
			\multicolumn{2}{|l|}{\textbf{Class Name: Show Subsystem Listener}} 
			\\
			\hline
			\textbf{Responsibility:} & \textbf{Collaborators:} \\
			\hline
			Listen for Show Subsystem taps from the GUI & activity\_main\\
			\vspace{0.2in} & \\
			\hline
			Inform inner system when a Subsystem has been shown/hidden &
			Fred Attribute Data Store\\ 
			\vspace{0.2in} & \\
			\hline
		\end{tabular}
	\end{table}
	%Class 10
	
	
	
	
	
	
	
	
	
	\begin{table}[ht]
		\centering
		\begin{tabular}{|p{5cm}|p{5cm}|}
		\hline 
		 \multicolumn{2}{|l|}{\textbf{Class Name: Random Event Generator}} \\
		\hline
		\textbf{Responsibility:} & \textbf{Collaborators:} \\
		\hline
		Generate random numbers to trigger events independent of User input & 
		Event Data Store\\
		\vspace{0.2in} & \\
		\hline
		\end{tabular}
	\end{table}
	%Class 12
	\begin{table}[ht]
		\centering
		\begin{tabular}{|p{5cm}|p{5cm}|}
		\hline 
		 \multicolumn{2}{|l|}{\textbf{Class Name: Fred Attribute Net Result}} \\
		\hline
		\textbf{Responsibility:} & \textbf{Collaborators:} \\
		\hline
		Determine the net effect on the system from a stimulus & Stimuli Listener\\
		\vspace{0.2in} & \\
		\hline
		Change Fred's state to match the effect of the stimulus & Fred Attribute Data Store\\
		\vspace{0.4in} & \\
		\hline
		\end{tabular}
	\end{table}
	%Class 13
	\begin{table}[ht]
		\centering
		\begin{tabular}{|p{5cm}|p{5cm}|}
		\hline 
		 \multicolumn{2}{|l|}{\textbf{Class Name: Fred Attribute Data Store}} \\
		\hline
		\textbf{Responsibility:} & \textbf{Collaborators:} \\
		\hline
		Receive changes to Fred's state from a change in the stimulus & Fred Attribute Net Result Calc\\
		\vspace{0.2in} & \\
		\hline
		Toggle (ON or OFF) the graphical view of a subsystem & Show Subsystem Listener\\
		\vspace{0.2in} & \\
		\hline
		Toggle (ON or OFF) a subsystem & Toggle Subsystem Listener\\
		\vspace{0.1in} & \\
		\hline
		Send Fred's updated state to the controller to update the view & MainActivity\\
		\end{tabular}
	\end{table}
	%Class 14
	\begin{table}[ht]
		\centering
		\begin{tabular}{|p{5cm}|p{5cm}|}
		\hline 
		 \multicolumn{2}{|l|}{\textbf{Class Name: Event Data Store}} \\
		\hline
		\textbf{Responsibility:} & \textbf{Collaborators:} \\
		\hline
		Send data about the random events to the event generator & Random Event Generator\\
		\vspace{0.4in} & \\
		\hline
		\end{tabular}
	\end{table}
	
		\begin{table}[ht]
		\centering
		\begin{tabular}{|p{5cm}|p{5cm}|}
		\hline 
		 \multicolumn{2}{|l|}{\textbf{Class Name: MainActivity}} \\
		\hline
		\textbf{Responsibility:} & \textbf{Collaborators:} \\
		\hline
		Get updated state from Fred and prepare it for viewing & Fred Attribute Data Store\\
		\vspace{0.4in} & \\
		\hline
		Send updated GUI data to the presentation so that it can update the GUI & activity\_main\\
		\vspace{0.4in} & \\
		\hline
		\end{tabular}
	\end{table}

% End Section
\cleardoublepage
\appendix
\newpage
\section{Division of Labour}
\label{sec:division_of_labour}
% Begin Section

\begin{tabular}{l c r}
    \textbf{Name} & \textbf{Work Completed} & \textbf{Signature} \\
    
    Kunal Shah & Subsystems, Analysis Class Diagram & 
    \includegraphics[scale=0.2]{../Resources/Signature/Kunal-Sig.png} \\
    
    Gabriel Castagner & Introduction, System Architecture, Analysis Class Diagram &
    \includegraphics[scale=0.2]{../Resources/Signature/Gabe-Sig.png} \\
    
    Victor Velechovsky & Introduction & 
    \includegraphics[scale=0.3]{../Resources/Signature/Vic-Sig.png} \\
    
    Josh Mitchell & Subsystems, CRC Cards & 
    \includegraphics[scale=0.2]{../Resources/Signature/Josh-Sig.png} \\
\end{tabular}
% End Section



\end{document}
%------------------------------------------------------------------------------