\documentclass[]{article}

% Imported Packages
%------------------------------------------------------------------------------
\usepackage{amssymb}
\usepackage{amstext}
\usepackage{amsthm}
\usepackage{amsmath}
\usepackage{enumerate}
\usepackage{fancyhdr}
\usepackage[margin=1in]{geometry}
\usepackage{graphicx}
\usepackage{extarrows}
\usepackage{setspace}
%------------------------------------------------------------------------------

% Header and Footer
%------------------------------------------------------------------------------
\pagestyle{plain}  
\renewcommand\headrulewidth{0.4pt}                                      
\renewcommand\footrulewidth{0.4pt}                                    
%------------------------------------------------------------------------------

% Title Details
%------------------------------------------------------------------------------
\title{Deliverable \#2 Template}
\author{SE 3A04: Software Design II -- Large System Design}
\date{}                               
%------------------------------------------------------------------------------

% Document
%------------------------------------------------------------------------------
\begin{document}

\maketitle	

\section{Introduction}
\label{sec:introduction}
% Begin Section


\subsection{Purpose}
\label{sub:purpose}
% Begin SubSection

This document shall provide an brief overview of the overall Design Architecture
of FredPlusPlus.\\
The target audience of this document would be the clientele receiving the final project and any persons designated to implement the design of this product. The other target audience would be the TA's and peers that will inspect and/or mark this document.
% End SubSection

\subsection{System Description}
\label{sub:system_description}
% Begin SubSection
%\begin{enumerate}[a)]
	The system that is mentioned in the document is as system that replicates the anatomy system of the human body. The subsystems that the program will simulate will be the major body systems that were referred in Deliverable 1, see definitions for more information. The stimuli that acts upon the sub systems will directly affect other subsystems. For example, if the player fed Fred a fatty food, this would increase his BMI which would result in a higher blood pressure. The stimuli was applied to the Digestive system and resulted in a change in the Circulatory system.
	
	We have organized the system by grouping each body system into networks that have their own dependencies and *outsourced dependencies* from other networks that may or may not be from other body systems. Some networks will have stimuli that the player can directly affect, such as feeding, and some networks may occur indirectly from the player's actions or from stimuli generated by the system, these include a random chance to be hit by a car.
%\end{enumerate}
% End SubSection
\subsection{Definitions}
\label{sub:system_description}
% Begin SubSection

	Major Body Systems: The specific body functions that this term refers to are the Cardiovascular (Heart), Respiratory (Lungs), Gastrointestinal (Digestive), Locomotor (Musculoskeletal) and Nervous (Nerves and Brain) systems.

	Project: The project will use project, application, game, and piece of software interchangeably. These words all refer to the actual project that is being developed. 

	Networks: This refers to the major body systems that are featured in the project. An example of these networks being referenced would be "The high blood pressure is negativley affecting the Cardiovascular network."

	Stim Sys: Short for Stimulous system, this is refers to the stimuli that will directley affect Fred's numerical values. They include the sub systems referenced in 4.2, which are the Food System, Drink System, Medication System, Exercise System and Catastrophic Event System.


% End SubSection
\subsection{Overview}
\label{sub:overview}
% Begin SubSection
%\begin{enumerate}[a)]
	The rest of the document will contain the corresponding Use Case Diagrams, Analysis Class Diagrams,the System Architecture and the Class Responsibility Cards respectfully. They will visually demonstrate the user how the classes interact with one another, how the implemented architecture will represent the overall architecture, and the contents of the CRC Cards. It is recommended that readers inspect the document in sequential order to first understand the layout of the design, and then further understand the purpose of the CRC Cards.

%This includes Use Case Diagram, Analysis Class Diagram, System Architecture and Class Responsibility Cards.

%\end{enumerate}
% End SubSection

% End Section

\section{Use Case Diagram}
\label{sec:use_case_diagram}
% Begin Section
This section should provide a use case diagram for your application. 
\begin{enumerate}[a)]
	\item Each use case appearing in the diagram should be accompanied by a text description. 
\end{enumerate}
% End Section

\section{Analysis Class Diagram}
\label{sec:analysis_class_diagram}
% Begin Section
This section should provide an analysis class diagram for your application.
% End Section


\section{Architectural Design}
\label{sec:architectural_design}
% Begin Section
This section should provide an overview of the overall architectural design of your application. You overall architecture should show the division of the system into subsystems with high cohesion and low coupling.

\subsection{System Architecture}
\label{sub:system_architecture}
% Begin SubSection

	The overall Architecture that we will be using for this project will be the Presentation-Abstraction-Control or PAC design. This architecture’s components are decomposed in following sub sections:

\textbf{Presentation}: This section is responsible for handling all of the program’s GUI interfaces, images and user interfaces. It is dependent on the abstraction section for its specific information, it does not communicate backwards towards the abstraction section.

\textbf{Abstraction}: This section is responsible for all the input and output handling through the use of logic, triggers and data keeping. This section gets it’s inputs/ stimuli from the Controller section and then interprets it and when finished with it, sends the visualization of that data to the Presentation section. The bulk of the project will be put in this section.

\textbf{Control}: This section is responsible for handling all of the user input for the program and sending that information to the Abstraction section. It should be able to handle all the button handling, sets, input values without doing any calculations to them, only interpreting that info and sending it to the Abstraction to be used.

	Our program will use this architecture as it encourages high cohesion and low couponing through the use of modularizing the data and keeping one system responsible for logic calculations. The expected platform that we hope to implement this project on also encourages this as Android development platforms promote PAC architecture. Finally, the structure and organization of our project follows intimately follower the PAC architecture.

% End SubSection

\subsection{Subsystems}
\label{sub:subsystems}
% Begin SubSection

Fred's subsystems are represented by modular responses to stimuli.

\begin{enumerate}[1)]
	\item Food System
		\begin{enumerate}[a)]
			\item Stimuli include: Healthy food, Unhealthily food, and Spoiled food
		\end{enumerate}
	\item Drink System
		\begin{enumerate}[a)]
			\item Stimuli include: Water, Juice, and Beer
		\end{enumerate}
	\item Medication System
		\begin{enumerate}[a)]
			\item Stimuli is Medicine
		\end{enumerate}
	\item Exercise System
		\begin{enumerate}[a)]
			\item Stimuli include: Strength training, Cardio, Cross fit, and lazy day
		\end{enumerate}
	\item Catastrophic Event System
		\begin{enumerate}[a)]
			\item Stimuli include: Cancer, Hit by car, Falling down the stairs, Diarrhea, High/low blood pressure and pneumonia
		\end{enumerate}
\end{enumerate}
% End SubSection

% End Section
	
\section{Class Responsibility Collaboration (CRC) Cards}
\label{sec:class_responsibility_collaboration_crc_cards}
% Begin Section
This section should contain all of your CRC cards.

\begin{enumerate}[a)]
	\item Provide a CRC Card for each identified class
	\item Please use the format outlined in tutorial, i.e., 
	\begin{table}[ht]
		\centering
		\begin{tabular}{|p{5cm}|p{5cm}|}
		\hline 
		 \multicolumn{2}{|l|}{\textbf{Class Name:}} \\
		\hline
		\textbf{Responsibility:} & \textbf{Collaborators:} \\
		\hline
		\vspace{1in} & \\
		\hline
		\end{tabular}
	\end{table}
	
\end{enumerate}
% End Section

\appendix
\section{Division of Labour}
\label{sec:division_of_labour}
% Begin Section
Include a Division of Labour sheet which indicates the contributions of each team member. This sheet must be signed by all team members.
% End Section



\end{document}
%------------------------------------------------------------------------------